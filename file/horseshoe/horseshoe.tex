\documentclass[11pt, oneside]{article}   	% use "amsart" instead of "article" for AMSLaTeX format
\usepackage[margin=1in]{geometry}                		% See geometry.pdf to learn the layout options. There are lots.
\geometry{letterpaper}                   		% ... or a4paper or a5paper or ... 
%\geometry{landscape}                		% Activate for rotated page geometry
%\usepackage[parfill]{parskip}    		% Activate to begin paragraphs with an empty line rather than an indent
\usepackage{graphicx}				% Use pdf, png, jpg, or eps§ with pdflatex; use eps in DVI mode
								% TeX will automatically convert eps --> pdf in pdflatex		
\usepackage{amssymb}
\usepackage{mathtools}
\usepackage{amsthm}
\usepackage[usenames, dvipsnames]{color}
\usepackage{bm}
\usepackage{url}

\usepackage[onehalfspacing]{setspace}
\usepackage{natbib}
\usepackage{hyperref}

\usepackage{cancel}
\usepackage{float}
\usepackage{xcolor}

%\usepackage[demo]{graphicx}
\usepackage{subfig}

\def \bbeta {\bm{\beta}} 
\def \x {\mathbf{x}} 
\def \w {\mathbf{w}} 
\def \b {\mathbf{b}} 
\newcommand{\given}{\,|\,}


% Front matter
\title{Horseshoe for random features}
\author{Beau Coker}
\date{}							% Activate to display a given date or no date

\begin{document}
\maketitle


\section{Model}

Network:
\begin{align}
f(\x_n; \nu, \{\tau_d\}, \{\beta_k\}, \{\w_k\}, b ) &=\sum_{k=1}^K \beta_k \Phi(\x_n; \nu, \tau_d, \w_k, b ) \\
\Phi(\x_n; s_d, \w_k, b) &= \sqrt{2}\cos\left(  \sum_{d=1}^D \nu \tau_d w_{k,d} x_{n,d} \right)
\end{align}
In matrix notation:
\begin{align}
f(X; S, \bbeta) &= \Phi(X; S, W, b)\bbeta \\
\Phi(X; S, W, b) &= \sqrt{2}\cos(X S W + \b) \\
\end{align}
where $S=\text{diag}(\nu \tau_d,\dotsc,\nu \tau_D)$ and $W=[\w_1^T,\dotsc,\w_K^T]^T$.

Model:
\begin{align}
\beta_k &\sim \mathcal{N}(0, \sigma^2_\beta), \quad k=1,\dotsc,K \\
\tau_d &\sim C^+(0, b_\tau), \quad d=1,\dotsc,D \\
\nu &\sim C^+(0, b_\nu) \\
y_n \given \x_n, \{\beta_k\}, \nu, \{\tau_d\}, \{\w_k\}, b &\sim \mathcal{N}(f(\x_n; \nu, \{\tau_d\}, \{\beta_k\}), \sigma^2), \quad n=1,\dotsc,N 
\end{align}

To improve inference, we add auxiliary variables $\{ \lambda_d \}_{d=1}^D$ and $\vartheta$:
\begin{align}
p(\tau_d, \lambda_d) &= p(\tau_d \given \lambda_d) p(\lambda_d)= \text{InvGamma}\left(\tau_d;  \frac{1}{2}, \frac{1}{\lambda_d}\right)  \text{InvGamma}\left(\lambda_d; \frac{1}{2}, \frac{1}{b_\tau}\right)
\\
p(\nu, \vartheta) &= p(\nu \given \vartheta) p(\vartheta) = \text{InvGamma}\left(\nu;  \frac{1}{2}, \frac{1}{\vartheta}\right)  \text{InvGamma}\left(\vartheta; \frac{1}{2}, \frac{1}{b_\nu}\right)
\end{align}

\section{Variational approximation}
We approximate the posterior distribution of $\{\tau_d\}$, $\nu$, $\{\lambda_d\}$, and $\vartheta$ with a fully factorized variational distribution. We use a log normal distribution for $\nu$ and each $\tau_d$, and an inverse Gamma distribution for $\vartheta$ and each $\lambda_k$.

\section{Inference}

For $s=1,\dotsc$
\begin{itemize}
	\item Sample $\{\beta_k^{(s)} \}$ from the posterior conditional on $\nu$ and $\{\tau_d^{(s)}\}$ (conjugate).
	\item Update the variational parameters for $\{\tau_d\}$, $\nu$, $\{\lambda_d\}$, and $\vartheta$ by optimizing the ELBO conditional on $\{\beta_k^{(s)} \}$ for a fixed number of steps.
\end{itemize}

\end{document}  






